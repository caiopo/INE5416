\documentclass{article}
\usepackage[utf8]{inputenc}
\usepackage[portuguese]{babel}

\title{Paradigmas de Programação - Trabalho 3}
\author{Caio Pereira Oliveira e Salomão Rodrigues Jacinto}
\date{Dezembro de 2016}

\usepackage{verbatim}

\begin{document}

\maketitle

\section{Centróide}

\paragraph{}
O predicado recebe a imagem e retorna a posição do centróide, para isso ele
verifica todos os pontos que possuem valor 1 , ou seja, que são brancos na
imagem, calcula a soma de X e de Y destes pontos e divide pelo número de pontos
encontrados.
\paragraph{}
A soma é feita através do predicado calculateSum, o qual recebe uma lista de
pontos e vai acumulando a soma até a lista estar vazia.

\begin{verbatim}
?- centroidImg('img1.pgm', X, Y).
X = 17.5,
Y = 18.5.
\end{verbatim}

\section{Bordas}

\begin{verbatim}
\end{verbatim}

\section{Características}

\paragraph{}
As características retiradas do desenho são a média das distâncias dos pontos da
borda até o centróide e o desvio padrão destas distâncias. O predicado recebe a
imagem e utilizando os predicados dos itens anteriores de cálculo da borda e do
centróide verifica a distância de cada ponto da borda até o centróide calculando
a média. Essa média é a mesma utilizada para o cálculo do desvio padrão.

\begin{verbatim}
?- averageDistanceImg('img1.pgm', AvgD).
AvgD = 13.767007120492341.

?- standardDeviationImg('img1.pgm', Sd).
Sd = 0.26532033528755467.
\end{verbatim}

\section{Detector de Círculos}

\begin{verbatim}
\end{verbatim}

\end{document}
